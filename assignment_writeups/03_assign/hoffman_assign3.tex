\documentclass[11pt]{article}
\usepackage[margin = 1in]{geometry}
\usepackage{amsmath}
\usepackage{amssymb}
\usepackage{amsthm} % for proof environment
\usepackage{enumitem}
\usepackage{graphicx}
\usepackage{indentfirst}
\usepackage{caption}
\usepackage{subcaption}
\usepackage{lscape}
\usepackage{multirow}
\usepackage{array}
\usepackage{tikz}
\usetikzlibrary{calc} % for positioning tikz nodes
\usepackage[round]{natbib}

\renewcommand{\labelenumii}{\arabic*)}
\newcommand{\ev}{\mathbb{E}}
\newcommand{\inv}[1]{#1^{-1}}

\tikzset{%
	hollow/.style = {circle,draw,inner sep=1.5},
	solid/.style = {circle,draw,inner sep=1.5,fill=black}%
}


\begin{document}

\begin{flushleft}
	Nick Hoffman \\
	Game Theory, Spring 2020 A \\
	Assignment 3 \\
\end{flushleft}

\begin{enumerate}
	\item In the potential sale of $ B $ to $ A $, I first consider the strategy for firm $ B $. Firm $ B $ knows its value $ x $, and sets a minimum price $ m $, and the sale is completed if firm $ A $'s offer $ y $ is greater than or equal to $ m $. Because firm $ B $ receives a value of $ x $ if the sale is not completed, it is easy to see that their equilibrium strategy is to set $ m = x $. If firm $ B $ sets $ m < x $, then if firm $ A $ makes an offer such that $ m < y < x $, then the sale would be completed, but $ B $ would have been better off setting a higher $ m $ and receiving $ x $. Conversely, if $ B $ sets $ m > x $, then the sale would fail if firm $ A $ made an offer such that $ m > y > x $, but firm $ B $ would have been better off had the sale gone through. Thus, $ m = x $.
	
	Firm $ A $, then, aims to choose $ y $ to maximize its expected profit, knowing that $ m = x $ and $ x\sim \mathcal{U}[0,100] $. Firm $ A $'s payoff, knowing these pieces of information, is given by 
	\[u_A(y|x) = \begin{cases}
	\frac{3}{2}x - y & x \leq y \\
	0 & x > y
	\end{cases}\]
	Thus, their expected utility is given by 
	\begin{align*}
	\ev[u_A|x] &= \left(\frac{3}{2}x - y\right)\cdot\Pr(x \leq y) \\
	 &= \left(\frac{3}{2}x - y\right)\frac{y}{100} 
	\end{align*}
	The total expected payoff to firm $ A $ is given by 
	\[\int_{0}^{y}\left(\frac{3}{2}x - y\right) dx = \frac{3}{4}x^2 - xy \Biggr|_0^y = -\frac{y^2}{4}\]
	Differentiating the expected payoff with respect to $ y $ and setting equal to 0 yields
	\[-\frac{1}{2}y = 0 \implies y = 0 \]
	Thus, the optimal strategy for firm $ A $, given that firm $ B $ plays their equilibrium strategy of $ m = x $, is to bid $ y = 0 $. This seemingly counterintuitive result is a consequence of the payout structure, and the distribution of $ x $. 
	
	\item In the two-bidder mixed auction where each player $ i\in\{1,2\} $, the winner pays $ \alpha b_i + (1 - \alpha)b_{j} $, $ j\neq i $. Players' valuations $ v_1 $ and $ v_2 $ are both distributed $ \mathcal{U}[0,1] $. In general, expected utility for player $ i $ is given by 
	\[U_i(v_i, b_i, b_j) = \Pr(b_i > b_j) \cdot (v_i - \alpha b_i - (1 - \alpha) b_j ) + \frac{1}{2}\Pr(b_i = b_j)\cdot (v_i - \alpha b_i - (1 - \alpha) b_j )\]
	Because the distributions of the valuations are continuous, $ \Pr(b_i = b_j) = 0 $, and thus we can restrict attention to the first term in the sum. Assume that a symmetric bidding strategy exists, such that 
	\[b_i = \beta(v_i)\]
	with $ \beta(v_i) $ monotone increasing in $ v_i $. To begin, note that because $ \beta $ is monotone, $ \inv{\beta} $ exists. Then, 
	\begin{align*}
	\Pr(b_i > b_j) &= \Pr(b_i > \beta(v_j)) \\ &= \Pr(\inv{\beta}(b_i) > v_j) \\ &= F(\inv{\beta}(b_i)) \\
	&= \inv{\beta}(b_i)
	\end{align*}
	 As $ F $ is the uniform CDF on [0,1]. Thus, the expected payoff to player $ i $ is given by 
	 \[\int_{0}^{\inv{\beta}(b_i)} \big(v_i - \alpha b_i - (1 - \alpha) \beta(v_j) \big)dv_j \]
	 To optimize, I use Liebnitz's rule to compute 
	 \[\frac{\partial }{\partial b_i} \int_{0}^{\inv{\beta}(b_i)} \big(v_i - \alpha b_i - (1 - \alpha) \beta(v_j) \big)dv_j = \frac{v_i - b_i}{\beta^\prime(v_i)} - \alpha v_i  = 0\]
	The final equality can be transformed to an Ordinary differential equation of the form
	\[\beta^\prime(v_i) + \frac{1}{\alpha v_i} \beta(v_i) = \frac{1}{\alpha}\]
	The integrating factor is $  $
	\[\exp\left(\int \frac{1}{\alpha v_i} dv_i\right) = v_i^{\frac{1}{\alpha}}\]
	Multiplying both sides by the integrating factor, and noticing the product rule implicit in this form, gives 
	\[\frac{d}{d v_i}\beta(v_i)v_i^{\frac{1}{\alpha}} =  \frac{v_i^{\frac{1}{\alpha}}}{\alpha} \]
	Integrating both sides gives 
	\[\beta (v_i) = \frac{v}{\alpha + 1}\]
	This form implies that $ b_i = v_i $ if $\alpha = 0 $, and $ b_i = v_i/2 $ when $ \alpha = 1 $, which are the forms that we know for first and second-price auctions. Furthermore, this form shows that $ \beta (0) = 0 $, and $ \beta $ is increasing in $ v $, which were the properties that we desired.  
	
	\item 
	\begin{enumerate}
		\item The game of incomplete information has the following form:
		\begin{itemize}
			\item Players $ i\in\{1,2\} $
			\item Types: $ \theta_1 $ fixed; player 1 does not know which game he is playing. $ \theta_2 \in\{t_2^1, t_2^2\}$; player 2 knows which game they are playing. 
			\item Beliefs: player one assigns equal $ 1/2 $ probability to $ t_2^1 $ and $ t_2^2 $. 
			\item Strategies: $ s_1: \Theta_1 \to \{a_1, a_2, a_3\} $, $ s_2: \Theta_2\to\{b_1, b_2\} $
			\item Payoffs given as follows: \newpage
			\begin{table}[!htbp]
				\centering
				\caption{$ G_1 $}
				\setlength{\extrarowheight}{2pt}
				\begin{tabular}{cc|c|c|c|}
					& \multicolumn{1}{c}{} & \multicolumn{1}{c}{$b_1$}  & \multicolumn{1}{c}{$b_2$}  \\\cline{3-4} 
					& $a_1$ & $ 3,0 $ & $ 0,1 $ \\\cline{3-4}
					& $a_2$ & $ 0,0 $ & $ 3,1 $ \\\cline{3-4}
					& $a_3$ & $ 2,0 $ & $ 2,1 $ \\\cline{3-4}
				\end{tabular}
			\end{table}
		
			\begin{table}[!htbp]
				\centering
				\caption{$ G_2 $}
				\setlength{\extrarowheight}{2pt}
				\begin{tabular}{cc|c|c|c|}
					& \multicolumn{1}{c}{} & \multicolumn{1}{c}{$b_1$}  & \multicolumn{1}{c}{$b_2$}  \\\cline{3-4} 
					& $a_1$ & $ 3,0 $ & $ 0,1 $ \\\cline{3-4}
					& $a_2$ & $ 0,1 $ & $ 3,0 $ \\\cline{3-4}
					& $a_3$ & $ 2,2 $ & $ 2,0 $ \\\cline{3-4}
				\end{tabular}
			\end{table}
		\end{itemize}
		
		In order to find the Bayesian Nash Equilibrium of this game, I note that player one chooses one strategy for both games, while player 2 chooses a strategy for $ G_1 $ and a strategy for $ G_2 $. Because each game occurs with probability $ 1/2 $, I combine them into one game as follows:
		\begin{table}[!htbp]
			\centering
			\setlength{\extrarowheight}{2pt}
			\begin{tabular}{cc|c|c|c|c|c|}
				& \multicolumn{1}{c}{} & \multicolumn{1}{c}{$b_1 b_1$}  & \multicolumn{1}{c}{$b_1 b_2$}  & \multicolumn{1}{c}{$b_2 b_1$} & \multicolumn{1}{c}{$b_2 b_2$}\\\cline{3-6} 
				& $a_1$ & $ 3,0 $ & $ \frac{3}{2}, \frac{1}{2} $ & $  \frac{3}{2}, \frac{1}{2} $ & $ 0,1 $ \\\cline{3-6}
				& $a_2$ & $ 0,\frac{1}{2} $ & $ \frac{3}{2}, 0 $ & $ \frac{3}{2}, 1 $ & $ 3, \frac{1}{2} $ \\\cline{3-6}
				& $a_3$ & $ 2,1 $ & $ 2,0 $ & $ 2, \frac{3}{2} $ & $ 2, \frac{1}{2} $ \\\cline{3-6}
			\end{tabular}
		\end{table}
	
		The Bayesian Nash Equilibrium for this game is thus $ P_1: a_3 $, $ P_2: b_2 b_1 $. 
	
		\item Now, the payoffs are the same, but the types are slightly different. The game of incomplete information has the following form:
		\begin{itemize}
			\item Players $ i\in\{1,2\} $
			\item Types: $ \theta_1\in\Theta_1 = \{t_1^1, t_1^2, t_1^3\} $. Type $ t_1^1 $ knows that the game is $ G_1 $, type $ t_1^2 $ knows that the game is $ G_2 $, and type $ t_1^3 $ does not know the game. For player 2, $ \theta_2\in\Theta_2 = \{t_2^1, t_2^2\} $ same as before.
			\item Beliefs: Player 2 assigns equal $ 1/2 $ probability to whether player 1 knows the game or not. If player 1 does not know the game, he assigns equal $ 1/2  $ probability to each game. 
		\end{itemize}
		
		The Bayesian Nash Equilibrium in pure strategies is defined as a set of strategies $ (s^*_i, s^*_{-i}) $ such that for all $ \theta $ and $ i $,
		\[s^*_i(\theta_i) = \arg\max_{s_i^\prime}\sum_{\theta_{-i}}\Pr(\theta_{-i}| \theta_i)u_i(s_i^\prime, s^*_{-i}, \theta) \]
		
		To begin, there is one elimination that we can consider: $ s_2^*(t_2^1) = b_2 $, as $ b_1 $ is a dominated strategy for player $ 2 $ in game 1. Knowing this, we can compute the following:\footnote{here, let $ u(G_j) $ denote the payoffs in game $ j $. }
		\begin{align*}
		s_1^*(t_1^1) &= \arg\max_a \Pr(t_2^1 | t_1^1)\cdot u(G_1) + \Pr(t_2^2 | t_1^1)\cdot u(G_2) \\
		&= \arg\max_a  u(G_1) + 0\cdot u(G_2) \\
		&= a_2 \\
		s_1^*(t_1^2) &= \arg\max_a \Pr(t_2^1 | t_1^2)\cdot u(G_1) + \Pr(t_2^2 | t_1^2)\cdot u(G_2) \\
		&= \arg\max_a 0\cdot u(G_1) + u(G_2) \\
		&= \emptyset\text{ as $ G_2 $ has no pure strategy Nash Equilibrium}
		\end{align*}
		To find the equilibrium strategies for the final two types, note that 
		\begin{align*}
		s_1^*(t_1^3) &= \arg\max_a \Pr(t_2^1 | t_1^3)u(G_1) + \Pr(t_2^2 | t_1^3)u(G_2) \\
		&= \frac{1}{2}u(G_1) + \frac{1}{2}u(G_2)
		\end{align*}
		and
		\begin{align*}
		s_2^*(t_2^2) &= \arg\max_b \Pr(t_1^1 | t_2^2)u(G_1) + \Pr(t_1^1 | t_2^2)u(G_2) + \Pr(t_1^3|t_2^2)u(G_2) \\
		&= u(G_2)
		\end{align*}
		Thus, in determining these optimal strategies, player 1 considers a mix of half of the payoffs from $ G_1 $ and half of the payoffs from $ G_2 $, while player 2 considers only the payoffs from $ G_2 $. Thus, the game is identical to $ G_2 $, and thus has no Nash Equilibrium in pure strategies, and so  $ s_1^*(t_1^3) $ and $ s_2^*(t_2^2) $ are empty. 
	\end{enumerate}
	\item 
	\begin{enumerate}
		\item In this case of bilateral trade, the buyer offers $ p_b $, while the seller posts $ p_s $. If $ p_s \leq p_b $, then the item sells for price $ p = \frac{1}{2}(p_s + p_b) $. Assume that both players play a linear equilibrium strategy, that is,
		\[p_i = \alpha_i + \beta_i v_i\]
		for $ i\in\{b, s\} $. In order to calculate the exact equilibrium strategy, I first formulate the best response for the buyer and the seller.
		
		The buyer's payoff is $ v_b - \frac{1}{2}(p_s + p_b) $ if $ p_s \leq p_b $, and 0 otherwise. Assuming that the seller follows $ p_s = \alpha_s + \beta_s v_s $, 
		
		\[\Pr(p_s \leq p_b) = F\left(\frac{p_b - \alpha_s}{\beta_s}\right)\]
		
		where $ F $ is the uniform CDF on [0,1]. Thus, the buyer's expected utility is given by 
		\begin{align*}
		\ev[u_b] &= \int_0^{\frac{p_b - \alpha_s}{\beta_s}} \left(v_b - \frac{1}{2}(p_s + p_b)\right) dv_s\\
		&= \int_0^{\frac{p_b - \alpha_s}{\beta_s}} \left(v_b - \frac{1}{2} (p_b + \alpha_s + \beta_s v_s)\right)dv_s
		\end{align*}
		To optimize, using Liebniz's rule, I compute
		\[\frac{\partial}{\partial p_b} \int_0^{\frac{p_b - \alpha_s}{\beta_s}} \left(v_b - \frac{1}{2} (p_b + \alpha_s + \beta_s v_s)\right)dv_s \]
		and set the result equal to 0 to obtain  
		\[p_b = \frac{2}{3}v_b + \frac{\alpha_s}{3}\]
		
		The seller's payoff, meanwhile, is $ \frac{1}{2}(p_s + p_b) - v_s$ if $ p_s \leq p_b $, and $ v_s $ otherwise. Thus, using the same procedure as for the buyer, I calculate
		\[\frac{\partial}{\partial p_s}\int_0^{\frac{p_s - \alpha_b}{\beta_b}} v_s dv_b + \frac{\partial}{\partial p_s}\int_{\frac{p_s - \alpha_b}{\beta_b}}^1 \left(\frac{1}{2}(p_s + p_b) - v_s\right) dv_b \]
		Setting this result equal to 0 and solving for $ p_s $ yields
		\[p_s = \frac{4}{3} v_s + \frac{\beta_b + 2\alpha_b}{3}\]
		To solve for the coefficients, I impose that $ p_b = \alpha_b + \beta_b v_b $ and $ p_s = \alpha_s + \beta_s v_s $:
		\begin{align*}
		\alpha_b + \beta_b v_b &= \frac{2}{3}v_b + \frac{\alpha_s}{3} \\
		\implies\beta_b = \frac{2}{3}&, \quad\alpha_b = \frac{\alpha_s}{3} \\
		\alpha_s + \beta_s v_s &=  \frac{4}{3} v_s + \frac{\beta_b + 2\alpha_b}{3} \\
		\implies\beta_s = \frac{4}{3}&, \quad \alpha_s = \frac{\frac{2}{3} + 2\alpha_b}{3}
		\end{align*}
		solving yields $ \alpha_b = 2 / 21 $ and $ \alpha_s = 6 / 21 $. Thus, the optimal strategies are 
		\begin{align*}
		p_b &= \frac{2}{3}v_b + \frac{2}{21} \\
		p_s &= \frac{4}{3}v_s + \frac{6}{21}
		\end{align*}
		
		\item Note that trade only takes place if $ p_b - p_s \geq 0 $, or if 
		\[\frac{2}{3}v_b + \frac{2}{21} - \frac{4}{3}v_s - \frac{6}{21} \geq 0 \]
		Simplifying, we see that this implies
		\[v_b - v_s \geq v_b - 2v_s \geq \frac{2}{7}\]
		Thus, trade does not always take place if $ v_b \geq v_s $, and thus this auction is not efficient. I found \cite{double} to be very useful for this problem. 
	\end{enumerate}
\end{enumerate}

\bibliographystyle{plainnat}
\bibliography{assign3}

\end{document}