\documentclass[11pt]{article}
\usepackage[margin = 1in]{geometry}
\usepackage{amsmath}
\usepackage{amssymb}
\usepackage{amsthm} % for proof environment
\usepackage{enumitem}
\usepackage{graphicx}
\usepackage{indentfirst}
\usepackage{caption}
\usepackage{lscape}

\renewcommand{\labelenumii}{\alph*)}

\begin{document}

\begin{flushleft}
	Nick Hoffman \\
	Game Theory, Spring 2020 A \\
	Assignment 1 \\
\end{flushleft}

\begin{enumerate}
	% 1)
	\item In the second-price auction, bidders have valuations $ v_i $ and submit bids $ b_i $. In this setup, the strategy of bidding one's valuation, $ b_i^* = v_i $, weakly dominates all other strategies.
	
	\begin{proof}
		Following the notation in Fudenberg and Tirole, let $ r_i = \max_{j\neq i} s_j $ denote the highest bid competing with $ s_i $. Generally speaking, player i has three strategies: $ s_i = v_i $, $ s_i < v_i $, and $ s_i > v_i $. 
		
		First, consider the strategy $ s_i > v_i $. If $ r_i > s_i > v_i $, then player $ i $ loses and gains utility 0, which is equivalent to what he would gain by bidding $ s_i = v_i $. Similarly, if $ r_i = s_i > v_i $, then they player wins with probability $ 1/2 $, in which case he gains utility $ v_i - r_i < 0 $, and loses with probability $ 1/2 $, and thus his expected utility is negative, leaving him worse off than if he had bid $ s_i = v_i $. If $ r_i \leq v_i < s_i $, then player $ i $ wins, but gains utility $ v_i - r_i > 0 $, which is exactly the benefit to playing $ s_i = v_i $. Lastly, if $ v_i < r_i < s_i $, then the player wins, but gains $ v_i - r_i < 0 $, and thus this strategy is dominated by bidding $ s_i = v_i $.
		
		Now, consider the strategy $ s_i < v_i $. If $ s_i < v_i \leq r_i $, then player $ i $ loses and gains utility 0, which he would have gained if he had bid $ s_i = v_i $. If $ s_i < r_i \leq v_i $, then player $ i $ loses and gains nothing, and thus would have been better off bidding $ s_i = v_i $, a strategy which would have positive expected utility. Lastly, if $ r_i \leq s_i < v_i $, then player $ i $ wins and gains $ v_i - r_i $, which is equivalent to the benefit to bidding $ s_i = v_i $.
		
		Thus, both strategies $ s_i < v_i $ and $ s_i > v_i $ are weakly dominated by strategy $ s_i = v_i $. 
	\end{proof} 

	% 2)
	\item In the travelers' dilemma, players $ i\in \{1,2\} $ submit claims $ c_i $, with $ c_i $ an integer such that $ 180 \leq c_i \leq 300 $. 
	\begin{enumerate}
		\item The process of elimination of weakly dominated strategies can be defined as follows. The initial set of strategies is given by 
		\[S_i^0 = \{180, 181, \dots, 300 \} \]
	\end{enumerate}
\end{enumerate}

\end{document}