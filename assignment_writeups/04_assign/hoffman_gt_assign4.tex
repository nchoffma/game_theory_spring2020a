\documentclass[11pt]{article}
\usepackage[margin = 1in]{geometry}
\usepackage{amsmath}
\usepackage{amssymb}
\usepackage{amsthm} % for proof environment
\usepackage{enumitem}
\usepackage{graphicx}
\usepackage{indentfirst}
\usepackage{caption}
\usepackage{subcaption}
\usepackage{lscape}
\usepackage{multirow}
\usepackage{array}
\usepackage{tikz}
\usetikzlibrary{calc} % for positioning tikz nodes

\renewcommand{\labelenumii}{\alph*)}
\newcommand{\ev}{\mathbb{E}}

%\tikzset{%
%	hollow/.style = {circle,draw,inner sep=1.5},
%	solid/.style = {circle,draw,inner sep=1.5,fill=black}%
%}

\begin{document}

\begin{flushleft}
	Nick Hoffman \\
	Game Theory, Spring 2020 A \\
	Assignment 4 \\
\end{flushleft}

\begin{enumerate}
	\item In the job market signaling model, workers are of type $ \theta\in\{\theta_L, \theta_H\} = \{0.2, 1\} $, each with equal one-half probability. In order to obtain education level $ e $ as a signal to the firms, the workers pay cost $ e/\theta $. The firm sets its wages such that $ w(e) = \ev(\theta|e) $. Thus, a worker of type $\theta$ receives utility $ w(e) - \frac{e}{\theta} $. The firms form beliefs of the type $ \mu(\theta|e) $, the probability that the worker is of type $\theta$ given her level of education. 
	\begin{enumerate}
		\item In a separating equilibrium, workers of each type choose different levels of education $ e^* $, where $ e^*(\theta_L) < e^*(\theta_H) $. For simplicity, I denote these equilibrium levels of education $ e_L $ and $ e_H $, respectively. The firms, knowing this, can differentiate, and thus for the low worker, they set
		\[w(e_L) = \ev(\theta_L|e_L) = \mu(e_L)\theta_L + \big(1 - \mu(e_l)\theta_H\big) = \theta_L\]
		The low-type worker then solves
		\[\max_e \theta_L - \frac{e_L}{0.2}\] 
		and thus chooses $ e_L = 0 $. 
		
		The firm can set a range of wage schedules $ w(e) $ to support a range of education levels for the high type. As with the low type, the firm will set $ w(e_H) = \theta_H = 1 $. What remains to be determined is the level of education at or above which a worker can earn the wage of 1. To show the result visually, Figure () plots indifference curves for the high and low types:
		
		The firm can sustain any level of education for the high type $ e_H\in[0.16, 0.8] $ by offering $ w(e) = 1 $ for any level of education $ e\in[e_H, \infty] $. If they attempt to sustain $ e_H < 0.16 $, then the high-type workers are better off getting zero education. If the firm attempts to sustain $ e_H > 0.8 $, meanwhile, the low-type workers will also choose $ e_H $, and thus this level cannot be a separating equilibrium. Thus, the  equilibrium which sustains $ e_L = 0 $ and $ e_H = \hat{e} $ is characterized by firm beliefs
		\[\mu(\theta_H|e) = \begin{cases}
		0 & e\in[0, \hat{e}) \\ 
		1 & e\in[\hat{e}, \infty)
		\end{cases}\]
		and wage schedule 
		\[w(e) = \begin{cases}
		0.2 & e\in[0, \hat{e}) \\ 
		1 & e\in[\hat{e}, \infty)
		\end{cases}\]
		
		\item In the pooling equilibrium, both types of workers choose the same level of education, the firm sets one wage, equal to the unconditional expectation of $\theta$:
		\[w(e) = 0.5\theta_L + 0.5\theta_H = 0.6 \]
		Figure () shows this wage along with indifference curves for the high and low type workers:
		
		The firm can sustain any pooling equilibrium level of education $ \hat{e}^*\in[0, 0.08] $. If they set $ \hat{e} > 0.08 $, the low-type workers are better off choosing $ e_L = 0 $. The pooling equilibrium that sustains $ \hat{e}^*\in[0, 0.08] $ is fully characterized by firm beliefs
		\[\mu(e) = \begin{cases}
		0 & e\in[0, \hat{e}) \\ 
		\frac{1}{2} & e\in[\hat{e}, \infty)
		\end{cases}\]
		and wage schedule
		\[w(e) = \begin{cases}
		0.2 & e\in[0, \hat{e}) \\ 
		0.6 & e\in[\hat{e}, \infty)
		\end{cases}\]
	\end{enumerate}

	\item The tree for the extensive form of the game is as follows:
	% https://sites.google.com/site/kochiuyu/Tikz#TOC-Signalling-Game
	
	There are no separating or pooling Perfect Bayesian equilibria in pure strategies. In order to find the mixed-strategy PBE, I define variables for the players' respective strategies. Let $ T_R $ and $ L_R $ denote Anna's actions, respectively, of taking and leaving her umbrella when it rains, and $ T_S $ and $ L_S $ the actions of taking and leaving when it is sunny. Anna's corresponding mixed strategies are $ \gamma_R $ and $\gamma_S$, indicating the probability of her taking her umbrella when it rains and is sunny, respectively. Note that because he cannot observe the weather forecast, Paul observes either $ (T_R, T_S) $ or $ (L_R, L_S) $. Thus, let $ \delta $ be the probability that Paul takes his umbrella given that he observes Anna leave hers, and $ \eta $ be the probability that he takes his umbrella given that Anna has taken hers. Finally, let $\alpha$ be the probability that Paul assigns to it raining when he observes $ L $, and $\beta$ be the probability that he assigns to it raining when he observes $ T $. The following diagram illustrates these probabilities:
	
	Paul's utilities from responding to Anna's actions are as follows:
	\begin{align*}
	\text{Observe L: }\quad &U(T) = 6\alpha - 2 \\
	& U(L) = 2 - 4\alpha \\
	\text{Observe T: } \quad &U(T) = 2\beta - 2 \\
	&U(L) = 2 - 4\beta
	\end{align*}
	Thus, his best responses are as follows: 
	\[BR(\alpha) = \begin{cases}
	T & \alpha > \frac{2}{5} \\
	[T,L] & \alpha = \frac{2}{5} \\
	L & \alpha < \frac{2}{5} \\
	\end{cases} \]%
	\[BR(\beta) = \begin{cases}
	T & \beta > \frac{2}{3} \\
	[T,L] & \beta = \frac{2}{3} \\
	L & \beta < \frac{2}{3} \\
	\end{cases} \]
	
	To begin, note that $ \gamma_R = \gamma_S = 1 $ cannot be an equilibrium: in this case, Paul updates his beliefs in the following way: 
	\[\beta = \frac{0.4\gamma_R}{0.4\gamma_R + 0.6\gamma_S} = 0.4\]
	By the above, Paul plays $ L $. $\alpha$, then, must be $ 2/5 $, otherwise, these would be pure strategies, which do not constitute an equilibrium in this case. Here, however, there is no value for $\delta$ such that Amy is still better off playing $ (T_1, T_2) $. Thus, this cannot be an equilibrium.
	
	If Anna plays $ \gamma_R  = \gamma_S = 0 $, then Paul will update his beliefs in the following way:
	\[\alpha = \frac{0.4}{0.4 + 0.6} = 0.4 = \frac{2}{5}\]
	and thus he will randomize between $ T $ and $ L $. As before, $\beta$ must be $ 2/3 $, otherwise this would not be an equilibrium. To determine Paul's strategy, note that Anna's best responses at nodes $ x_1 $ and $ x_2 $ are:
	\begin{align*}
	x_1: \quad & U(T_1) = 3 - 3\eta \\
	&U(L_1) = 1 - 3\delta \\
	x_2:\quad & U(T_2) = 4\eta - 1 \\
	& U(L_1) = 3\delta + 2
	\end{align*}
	Thus, Anna will be better off playing $ (L_1, L_2) $ if Paul sets his strategy such that, given $\eta$, 
	\[\delta\in[\frac{3}{4}\eta - \frac{3}{4}, \eta - \frac{2}{3}]\]
	Thus, the Perfect Bayesian Equilibrium is as follows: 
	\begin{itemize}
		\item $ P_1 $: $ \gamma_R = \gamma_S = 0 $ $ (L_1, L_2) $
		\item \[P_2: \begin{cases}
		\eta T \oplus (1 - \eta)L & \text{if $ (T_1, T_2) $} \\
		\delta T \oplus (1 - \delta) L & \text{if $ (L_1, L_2) $}
		\end{cases}\]
		where 
		\[\delta\in[\frac{3}{4}\eta - \frac{3}{4}, \eta - \frac{2}{3}]\]
		\item Beliefs: $\alpha = 2/5$, $ \beta = 2/5 $ 
	\end{itemize}

	Note that this is the only PBE: if we consider $ \gamma_R\in(0,1) $ and $ \gamma_S\in(0,1) $, these proportions must be such that $\alpha = 2/5$ and $ \beta = 2/5 $. That is, they must be such that 
	\begin{align*}
	\frac{0.4\gamma_R}{0.4\gamma_R + 0.6\gamma_S} &= \frac{2}{3} \text{, and}\\
	\frac{0.4(1 - \gamma_R)}{0.4(1 - \gamma_R) + 0.6(1 - \gamma_S)} &= \frac{2}{5} 
	\end{align*}
	There is no combination of $ \gamma_R\in(0,1) $ and $ \gamma_S\in(0,1) $ that satisfies both of the above equations. Thus, the PBE specified above is unique. 
	
	\item	
\end{enumerate}

\end{document}